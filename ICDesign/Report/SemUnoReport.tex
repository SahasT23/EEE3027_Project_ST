\documentclass[a4paper,12pt]{article}

\usepackage{url}
\usepackage{parskip} 	
\usepackage{tabularx}

%other packages for formatting
\RequirePackage{color}
\RequirePackage{graphicx}
\usepackage[usenames,dvipsnames]{xcolor}
\usepackage[scale=0.9]{geometry}
\usepackage{amsmath}
\usepackage{tikz}
\usetikzlibrary{shapes,arrows}
\usepackage{rotating}
\usetikzlibrary{shapes.geometric, arrows}
\usetikzlibrary{positioning, shapes}
\usepackage{float}
\usepackage{algorithm}
\usepackage{algpseudocode}
\usepackage{tikz}
\usetikzlibrary{arrows.meta, positioning}
\usepackage{setspace}
\usepackage{titlesec}
\usepackage{tikz}
\usepackage{tabularx}
\usepackage{circuitikz}
\usepackage{pgfplots}
\usepackage{amssymb}
\usepackage{amsthm}
\usepackage{float}  
\usepackage{booktabs}
\usepackage{xcolor}
\pgfplotsset{compat=1.18}
\usepackage{setspace}
\usepackage{colortbl}
\usepackage{tcolorbox}
\usepackage{wrapfig}
\usepackage{media9}
\usepackage{siunitx}

% Define colors
\definecolor{blueblock}{RGB}{173, 216, 230}  % Light blue
\definecolor{greenblock}{RGB}{144, 238, 144} % Light green
\definecolor{redblock}{RGB}{255, 182, 193}   % Light red
\definecolor{lightblue}{RGB}{173,216,230}
\definecolor{lightgreen}{RGB}{144,238,144}
\definecolor{lightorange}{RGB}{255,200,150}
\definecolor{lightpurple}{RGB}{220,180,240}

\tikzstyle{startstop} = [rectangle, rounded corners, minimum width=3cm, minimum height=1cm,text centered, draw=black, fill=red!30]
\tikzstyle{process} = [rectangle, minimum width=3cm, minimum height=1cm, text centered, draw=black, fill=blue!30]
\tikzstyle{decision} = [diamond, minimum width=3cm, minimum height=1cm, text centered, draw=black, fill=green!30]
\tikzstyle{arrow} = [thick,->,>=stealth]
\tikzstyle{io} = [trapezium, trapezium left angle=70, trapezium right angle=110, minimum width=3cm, minimum height=1cm, text centered, draw=black, fill=blue!30]

% Define styles with smaller dimensions and font size
\tikzset{
    font=\small,
    startstop/.style={
        rectangle,
        rounded corners,
        minimum width=2.5cm,
        minimum height=0.8cm,
        draw=black,
        fill=red!30,
        text centered,
        % Set text width to wrap text automatically
        text width=3.0cm,
        align=center
    },
    process/.style={
        rectangle,
        minimum width=2.5cm,
        minimum height=0.8cm,
        draw=black,
        fill=orange!30,
        text centered,
        % Set text width to wrap text automatically
        text width=3.0cm,
        align=center
    },
    decision/.style={
        diamond,
        aspect=2,
        draw=black,
        fill=green!30,
        inner sep=0pt,
        minimum width=2.0cm,
        minimum height=1.0cm,
        % Allow text wrapping in decision blocks
        text width=2.2cm,
        align=center
    },
    io/.style={
        trapezium,
        trapezium left angle=70,
        trapezium right angle=110,
        minimum width=2.5cm,
        minimum height=0.8cm,
        draw=black,
        fill=blue!30,
        text centered,
        % Set text width to wrap text automatically
        text width=3.0cm,
        align=center
    },
    arrow/.style={
        thick,
        ->,
        >=Stealth,
        shorten <=1pt,
        shorten >=1pt
    },
}

\definecolor{codegreen}{rgb}{0,0.6,0}
\definecolor{codegray}{rgb}{0.5,0.5,0.5}
\definecolor{codepurple}{rgb}{0.58,0,0.82}
\definecolor{backcolour}{rgb}{0.95,0.95,0.92}

\usepackage{listings}

% Define C style
\lstdefinestyle{cstyle}{
    backgroundcolor=\color{backcolour},
    commentstyle=\color{codegreen},
    keywordstyle=\color{blue}, % Keywords in blue
    numberstyle=\tiny\color{codegray},
    stringstyle=\color{codepurple},
    basicstyle=\ttfamily\footnotesize,
    breakatwhitespace=false,
    breaklines=true,
    captionpos=b,
    keepspaces=true,
    numbers=left,
    numbersep=5pt,
    showspaces=false,
    showstringspaces=false,
    showtabs=false,
    tabsize=4,
    language=C,
    morekeywords={uint32_t, uint64_t, int32_t, int64_t, bool, inline}
}

\lstset{style=cstyle}

% Page layout settings
\geometry{a4paper, margin=1in}

%tabularx environment
\usepackage{tabularx}

%for lists within experience section
\usepackage{enumitem}

% centered version of 'X' col. type
\newcolumntype{C}{>{\centering\arraybackslash}X} 

%to prevent spillover of tabular into next pages
\usepackage{supertabular}
\usepackage{tabularx}
\newlength{\fullcollw}
\setlength{\fullcollw}{0.47\textwidth}

%custom \section
\usepackage{titlesec}				
\usepackage{multicol}
\usepackage{multirow}

%CV Sections inspired by: 
%http://stefano.italians.nl/archives/26
\titleformat{\section}{\large\scshape\raggedright}{}{0em}{}[\titlerule]
\titlespacing{\section}{0pt}{10pt}{10pt}

%for publications
\usepackage[style=authoryear,sorting=ynt, maxbibnames=2]{biblatex}

%Setup hyperref package, and colours for links
\usepackage[unicode, draft=false]{hyperref}
\definecolor{linkcolour}{rgb}{0,0.2,0.6}
\hypersetup{colorlinks,breaklinks,urlcolor=linkcolour,linkcolor=linkcolour}
\addbibresource{citations.bib}
\setlength\bibitemsep{1em}

%for social icons
\usepackage{fontawesome5}

\begin{document}
%----------------------------------------------------------------------------------------
%	TITLE+
%----------------------------------------------------------------------------------------
% Title Page
\begin{titlepage}
    \centering
    \vspace*{2cm}
    \Huge{\textbf{EEE3027 Integrated Circuit Design}}\\[0.5cm]
    \Large{\textbf{Semester 1 Report}}\\ 
    \Large{Sahas Talasila \textit{230057896}}
    \vfill
\end{titlepage}

% Table of Contents
\tableofcontents
\newpage

\begin{abstract}
    This is a placeholder for the abstract of the report.
\end{abstract}

\section{Introduction}

\subsection{Background}
This is a placeholder for the background information.

\begin{itemize}
    \item Lab 1 Part 1
    \item N-Type characteristics
    \item P-Type characteristics
    \item MOSFET operation
    \item Parts that constitute a MOSFET and what altering them does to the operation of the MOSFET.
    \item Explanation of the PMOS and NMOS operation regions.
    \item What measurements were taken and how they were taken.
    \item Explanation of the graphs plotted and what they signify.
    \item Include code snippets of the simulation code used.
    \item Include graphs plotted.
    \item Analysis of the results.
    \begin{enumerate}
        \item DC Analysis of NMOS and PMOS and explanation of what DC analysis is.
        \item Task 1: NMOS DC I-V sweep with annotations.
        \item Task 2: PMOS DC I-V sweep with annotations.
        \item Gain factor calculation and explanation.
        \item Beta explanation with Python example.
        \item Task 4 Iterative solution
    \end{enumerate}
    \item Lab 1 Part 2
    \item Explain what static analysis is.
    \item Draw an annotated schematic of the inverter circuit.
    \item Explain the operation of the inverter circuit.
    \item 1B, Annotate the file (All 7 lines)
    \item TASK 1C: DC Modes You have simulated the DC transfer characteristic of a CMOS inverter by sweeping the input voltage Vin from 0 V to 5 V. Plot the 
    output voltage Vout against Vin. On your plot of Vout vs. Vin, clearly mark and label the following five operating points:
    A: Vin=0, B: Vin=1, C: Vin=2.5, D: Vin=4, E: Vin=5, For each point (A–E), identify the operating region of both the NMOS and PMOS transistors: Cutoff
    Triode (Linear) or Saturation, showing it as a table, and approximate Vout at each point.
    \item TASK 1D, NMOS pull up example, show the sweep as well, then answer these 5 questions:
    \item Q1. (1 mark) What is the highest output voltage reached in the NMOS pull-up version? Why does it not reach 5 V?
    \item Q2. (1 mark) Explain what happens to the NMOS pull-up transistor when Vin=0. Is it conducting?
    \item Q3. (1 mark) Describe what would happen if this NMOS-only inverter drove another CMOS logic gate. What are the risks?
    \item Q4. (1 mark) Why does a PMOS transistor avoid this issue? How does it behave differently from NMOS in pull-up?
    \item Q5. (2 marks) On your NMOS-only inverter plot, annotate the output voltage limit and mark the region where a 1 cannot be produced.
    \item TASK 2: TASK 2: CMOS Inverter Design for Delay Specification
    Use LTspice and theoretical models to design an inverter that meets timing specifications. Analyze
    performance using simulation and transistor-level delay theory (including propagation delay, rise/fall
    time, capacitive loading).
    \item TASK 2 Question 1 Transient Simulation and Analysis.
    \item TASK 2 Question 2 RC Delay Model.
    \item TASK 2 Question 3 Design and New Alternatives for performance.
    \item TASK 2 Question 4 Selection of questions to be answered.
\end{itemize}

\end{document}